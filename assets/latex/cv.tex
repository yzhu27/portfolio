\documentclass[10pt, letterpaper]{article}

% Packages:
\usepackage[
    ignoreheadfoot, % set margins without considering header and footer
    top=1 cm, % seperation between body and page edge from the top
    bottom=1 cm, % seperation between body and page edge from the bottom
    left=2 cm, % seperation between body and page edge from the left
    right=2 cm, % seperation between body and page edge from the right
    footskip=1.0 cm, % seperation between body and footer
    % showframe % for debugging 
]{geometry} % for adjusting page geometry
\usepackage{titlesec} % for customizing section titles
\usepackage{tabularx} % for making tables with fixed width columns
\usepackage{array} % tabularx requires this
\usepackage[dvipsnames]{xcolor} % for coloring text
\definecolor{primaryColor}{RGB}{0, 79, 144} % define primary color
\usepackage{enumitem} % for customizing lists
\usepackage{fontawesome5} % for using icons
\usepackage{amsmath} % for math
\usepackage[
    pdftitle={Yuheng Zhu's CV},
    pdfauthor={Yuheng Zhu},
    pdfcreator={LaTeX with RenderCV},
    colorlinks=true,
    urlcolor=primaryColor
]{hyperref} % for links, metadata and bookmarks
\usepackage[pscoord]{eso-pic} % for floating text on the page
\usepackage{calc} % for calculating lengths
\usepackage{bookmark} % for bookmarks
\usepackage{lastpage} % for getting the total number of pages
\usepackage{changepage} % for one column entries (adjustwidth environment)
\usepackage{paracol} % for two and three column entries
\usepackage{ifthen} % for conditional statements
\usepackage{needspace} % for avoiding page brake right after the section title
\usepackage{iftex} % check if engine is pdflatex, xetex or luatex

% Ensure that generate pdf is machine readable/ATS parsable:
\ifPDFTeX
    \input{glyphtounicode}
    \pdfgentounicode=1
    % \usepackage[T1]{fontenc} % this breaks sb2nov
    \usepackage[utf8]{inputenc}
    \usepackage{lmodern}
\fi



% Some settings:
\AtBeginEnvironment{adjustwidth}{\partopsep0pt} % remove space before adjustwidth environment
\pagestyle{empty} % no header or footer
\setcounter{secnumdepth}{0} % no section numbering
\setlength{\parindent}{0pt} % no indentation
\setlength{\topskip}{0pt} % no top skip
\setlength{\columnsep}{0cm} % set column seperation
\makeatletter
\let\ps@customFooterStyle\ps@plain % Copy the plain style to customFooterStyle
% \patchcmd{\ps@customFooterStyle}{\thepage}{
%     \color{gray}\textit{\small Yuheng Zhu - Page \thepage{} of \pageref*{LastPage}}
% }{}{} % replace number by desired string
\makeatother
% \pagestyle{customFooterStyle}

\titleformat{\section}{\needspace{4\baselineskip}\bfseries\large}{}{0pt}{}[\vspace{1pt}\titlerule]

\titlespacing{\section}{
    % left space:
    -1pt
}{
    % top space:
    0.1 cm
}{
    % bottom space:
    0.1 cm
} % section title spacing

\renewcommand\labelitemi{$\circ$} % custom bullet points
\newenvironment{highlights}{
    \begin{itemize}[
        topsep=0.10 cm,
        parsep=0.10 cm,
        partopsep=0pt,
        itemsep=0pt,
        leftmargin=0.4 cm + 10pt
    ]
}{
    \end{itemize}
} % new environment for highlights

\newenvironment{highlightsforbulletentries}{
    \begin{itemize}[
        topsep=0.10 cm,
        parsep=0.10 cm,
        partopsep=0pt,
        itemsep=0pt,
        leftmargin=10pt
    ]
}{
    \end{itemize}
} % new environment for highlights for bullet entries


\newenvironment{onecolentry}{
    \begin{adjustwidth}{
        0.2 cm + 0.00001 cm
    }{
        0.2 cm + 0.00001 cm
    }
}{
    \end{adjustwidth}
} % new environment for one column entries

\newenvironment{twocolentry}[2][]{
    \onecolentry
    \def\secondColumn{#2}
    \setcolumnwidth{\fill, 4.5 cm}
    \begin{paracol}{2}
}{
    \switchcolumn \raggedleft \secondColumn
    \end{paracol}
    \endonecolentry
} % new environment for two column entries

\newenvironment{header}{
    \setlength{\topsep}{0pt}\par\kern\topsep\centering\linespread{1.5}
}{
    \par\kern\topsep
} % new environment for the header

\newcommand{\placelastupdatedtext}{% \placetextbox{<horizontal pos>}{<vertical pos>}{<stuff>}
  \AddToShipoutPictureFG*{% Add <stuff> to current page foreground
    \put(
        \LenToUnit{\paperwidth-2 cm-0.2 cm+0.05cm},
        \LenToUnit{\paperheight-1.0 cm}
    ){\vtop{{\null}\makebox[0pt][c]{
        \small\color{gray}\textit{Last updated in September 2024}\hspace{\widthof{Last updated in September 2024}}
    }}}%
  }%
}%

% save the original href command in a new command:
\let\hrefWithoutArrow\href

% new command for external links:
\renewcommand{\href}[2]{\hrefWithoutArrow{#1}{\ifthenelse{\equal{#2}{}}{ }{#2 }\raisebox{.15ex}{\footnotesize \faExternalLink*}}}


\begin{document}
    \newcommand{\AND}{\unskip
        \cleaders\copy\ANDbox\hskip\wd\ANDbox
        \ignorespaces
    }
    \newsavebox\ANDbox
    \sbox\ANDbox{}

    % \placelastupdatedtext
    \begin{header}
        \textbf{\fontsize{20 pt}{20 pt}\selectfont Yuheng Zhu}

        \vspace{0.2 cm}

        \normalsize
        \mbox{{\color{black}\footnotesize\faMapMarker*}\hspace*{0.13cm}Raleigh, NC}%
        \kern 0.25 cm%
        \AND%
        \kern 0.25 cm%
        \mbox{\hrefWithoutArrow{mailto:yzhu63@ncsu.edu}{\color{black}{\footnotesize\faEnvelope[regular]}\hspace*{0.13cm}yzhu63@ncsu.edu}}%
        \kern 0.25 cm%
        \AND%
        \kern 0.25 cm%
        \mbox{\hrefWithoutArrow{tel:+1 (919)818-9960}{\color{black}{\footnotesize\faPhone*}\hspace*{0.13cm}(919)818-9960}}%
        \kern 0.25 cm%
        \AND%
        \kern 0.25 cm%
        \mbox{\hrefWithoutArrow{https://yuhengzhu.me/}{\color{black}{\footnotesize\faLink}\hspace*{0.13cm}yuhengzhu.me}}%
        \kern 0.25 cm%
        \AND%
        \kern 0.25 cm%
        \mbox{\hrefWithoutArrow{https://linkedin.com/in/yzhu27}{\color{black}{\footnotesize\faLinkedinIn}\hspace*{0.13cm}yzhu27}}%
        \kern 0.25 cm%
        \AND%
        \kern 0.25 cm%
        \mbox{\hrefWithoutArrow{https://github.com/yzhu27}{\color{black}{\footnotesize\faGithub}\hspace*{0.13cm}yzhu27}}%
    \end{header}

    \vspace{0.3 cm - 0.3 cm}


    % \section{Welcome to RenderCV!}



        
    %     \begin{onecolentry}
    %         \href{https://rendercv.com}{RenderCV} is a LaTeX-based CV/resume version-control and maintenance app. It allows you to create a high-quality CV or resume as a PDF file from a YAML file, with \textbf{Markdown syntax support} and \textbf{complete control over the LaTeX code}.
    %     \end{onecolentry}

    %     \vspace{0.2 cm}

    %     \begin{onecolentry}
    %         The boilerplate content was inspired by \href{https://github.com/dnl-blkv/mcdowell-cv}{Gayle McDowell}.
    %     \end{onecolentry}


    
    % \section{Quick Guide}

    % \begin{onecolentry}
    %     \begin{highlightsforbulletentries}


    %     \item Each section title is arbitrary and each section contains a list of entries.

    %     \item There are 7 unique entry types: \textit{BulletEntry}, \textit{TextEntry}, \textit{EducationEntry}, \textit{ExperienceEntry}, \textit{NormalEntry}, \textit{PublicationEntry}, and \textit{OneLineEntry}.

    %     \item Select a section title, pick an entry type, and start writing your section!

    %     \item \href{https://docs.rendercv.com/user_guide/}{Here}, you can find a comprehensive user guide for RenderCV.


    %     \end{highlightsforbulletentries}
    % \end{onecolentry}

    \section{Education}



        
        \begin{twocolentry}{
        \textit{Aug 2024 – Present}}
            \textbf{North Carolina State University}

            \textit{Ph.D. in Computer Science}
        \end{twocolentry}

        \begin{twocolentry}{
        \textit{Aug 2022 – May 2024}}
            \textbf{North Carolina State University}

            \textit{Master in Computer Science}
        \end{twocolentry}
        \begin{twocolentry}{
        \textit{Sept 2017 – May 2021}}
            \textbf{Southern University of Science and Technology}

            \textit{B.E. in Computer Science and Technology}
        \end{twocolentry}

        

        % \vspace{0.10 cm}
        % \begin{onecolentry}
        %     \begin{highlights}
        %         \item GPA: 4.0/4.0
        %         \item \textbf{Coursework:} Computer Architecture, Comparison of Learning Algorithms, Computational Theory
        %     \end{highlights}
        % \end{onecolentry}



    
    \section{Experience}

        \begin{twocolentry}{
        \textit{San Diego, CA}    
            
        \textit{Jun 2025 - Aug 2025}}
            \textbf{Engineer Intern}
            
            \textit{Qualcomm}
        \end{twocolentry}



        \vspace{0.1 cm}

        
        \begin{twocolentry}{
        \textit{Raleigh, NC}    
            
        \textit{Sept 2023 - Present}}
            \textbf{Research Assistant}
            
            \textit{North Carolina State University}
        \end{twocolentry}



        \vspace{0.1 cm}

        \begin{twocolentry}{
        \textit{Shenzhen, China}    
            
        \textit{June 2021 – July 2022}}
            \textbf{Research Assistant}
            
            \textit{Southern University of Science and Technology}
        \end{twocolentry}
        
        \vspace{0.1 cm}

        \begin{twocolentry}{
        \textit{Chengdu, China}    
            
        \textit{June 2020 – Aug 2020}}
            \textbf{Software Engineer Intern}
            
            \textit{JD.com, AI Research Center}
        \end{twocolentry}




    
    \section{Publications}



        
        \begin{samepage}
            \begin{onecolentry}
                \textbf{AdaptAV: Continuous Adaption of Vision Models for Autonomous Vehicles Using Cloud-based Oracle}. \mbox{\textbf{\textit{Yuheng Z.}}}, \mbox{Dhruva U.}, \mbox{Boluo G.} \& \mbox{Man-Ki Y.}, \textit{Proceedings of the 100th IEEE Vehicular Technology Conference}
            \end{onecolentry}
            \begin{onecolentry}
                \textbf{Bridging Data and Knowledge: A Neurosymbolic Framework for Reliable Network Analysis}. \mbox{Zhjin Y.}, \mbox{\textbf{\textit{Yuheng Z.}}}, \mbox{Mingzhe C.} \& \mbox{Yuchen L.}, \textit{Proceedings of 2025 IEEE Global Communications Conference} (Accepted)
            \end{onecolentry}
            \begin{onecolentry}
                \textbf{FrameScope: Temporal Data Valuation for Stream Active Learning in Autonomous Vehicle Systems}. \mbox{\textbf{\textit{Yuheng Z.}}}, \& \mbox{Man-Ki Y.} \textit{Proceedings of The 10th ACM/IEEE Symposium on Edge Computing} (Accepted)
            \end{onecolentry}


        \end{samepage}


    
    \section{Projects}





        \begin{twocolentry}{

        \textit{Qualcomm, 2025}}
            \textbf{Agentic LLM Testcase Triage Framework}
        \end{twocolentry}

        \vspace{0.10 cm}
        \begin{onecolentry}
            \begin{highlights}
                \item Independently built an agentic RAG testing failure analysis pipeline for 10K+ tests/day.
                \item Multi-source logs distilled via LLM summarization + error-anchor context, then chunked and indexed in a vector storage (ChromaDB). 
                \item Orchestrated with LangChain agents + function calling for hybrid retrieval (semantic + keyword) and top-k reranking, emitting JSON-schema reports.
                \item Tools used: \textbf{LangChain, ChromaDB, FAISS}
            \end{highlights}
        \end{onecolentry}


        \vspace{0.15 cm}
            
        \begin{twocolentry}{
            
            
        \textit{NCSU, 2023 - 2024}}
            \textbf{On-the-fly Coding of Vision Inputs for Evidence-Preserving Perception}
        \end{twocolentry}

        \vspace{0.10 cm}
        \begin{onecolentry}
            \begin{highlights}
                \item Designed and implemented a visual perception middleware for autonomous driving systems that preserves complete and reproducible evidence of visual inputs.
                \item Utilized compression techniques such as JPEG and H.264, with performance optimizations achieved through NVENC and NVDEC hardware acceleration engines. 
                \item Tools used: \textbf{C, Python, V4L2, , NVIDIA Jetpack, NVIDIA Codec, FFmpeg}
            \end{highlights}
        \end{onecolentry}

        \vspace{0.15 cm}

        \begin{twocolentry}{
            
            
        \textit{SUSTech, 2021 - 2022}}
            \textbf{Carla1s Distributed Online Simulation Platform}
        \end{twocolentry}

        \vspace{0.10 cm}
        \begin{onecolentry}
            \begin{highlights}
                \item Designed and developed a Carla-based remote and hardware-free web simulation platform.
                \item Developed a \textbf{backend streaming server} for \textbf{HTTP Streaming} \textbf{ROS} Image Topics and Carla Sensor Output, supporting multiple video encodings and streaming protocols, reduce latency of remote simulation from 1-2s to 50ms.
                \item Tools used: \textbf{Python Flask, FFmpeg, Ansible, Git, Docker}
            \end{highlights}
        \end{onecolentry}

        \vspace{0.15 cm}

        \begin{twocolentry}{
            
            
        \textit{JD.com, 2020}}
            \textbf{Jingxiaozhi AI Customer Service System}
        \end{twocolentry}

        \vspace{0.10 cm}
        \begin{onecolentry}
            \begin{highlights}
                \item Participated in two development cycles (2.1.6 \& 2.1.7) of JD Jingxiaozhi online AI customer service system, made in-depth research on the application and practice of Chinese \textbf{NLP} in e-commerce scenarios
                \item Developed web API automated testing modules in \textbf{Java} based on JD JSF \textbf{RPC framework}, made test framework can fully mimic user operation logic, increased API test coverage to 100\% 
                \item Tools used: \textbf{Java, RESTful API, JavaScript, Redis, RabbitMQ}
            \end{highlights}
        \end{onecolentry}

    
    \section{Technologies}



        
        \begin{onecolentry}
            \textbf{Languages:} C, C$^{++}$, Shell, Python, Java, JavaScript, Lua
        \end{onecolentry}

        \vspace{0.1 cm}

        \begin{onecolentry}
            \textbf{Skills:} Linux Kernel, ROS2, Networking, DevOps, Docker, Redis, Carla, ADAS, LangChain, ChromaDB
        \end{onecolentry}


    

\end{document}